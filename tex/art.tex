\documentclass[UTF8]{ctexart}

\begin{document}
假设两类博弈群体:政府和企业。
为简化起见,假定每次博弈都是企业群体中的一员与政府群体中的一员随机配对进行核心技术创新选择博弈,
并且每一群体中的一员都有两个纯策略选择。

\begin{itemize}
\item 企业可以选择进行核心技术创新和非核心技术创新。
\item 政府可以选择参与或不参与企业进行核心技术创新。
\end{itemize}

当企业采取核心技术创新策略时,如果政府选择参与支持企业进行核心技术创新,政府会对企业给予一定的补贴$A$。
企业因核心技术创新而获得预期收益为$\pi_1$。
政府给予核心技术创新企业监管投入记为$C_1$。
政府因支持企业核心技术创新而获得的声誉以及政府公信力的提升记为$b$。
企业获得政府对创新补助会向外释放积极信号,从而争取到更多的社会资源聚集企业(如外部投资的增加,上下游企业合作的增加等),
即核心技术创新带给企业的间接收益,记为$a$。

当企业选择非核心技术创新时所获得收益的同时,会受到声誉以及和政府合作机会的减少等损失$T$.
如果政府选择放任企业自己做出核心技术创新决策时,无论其核心技术创新
与否,政府均不承担核心技术创新的成本,政府也无创新收益。但若企业放弃核
心技术创新决策,则政府需要对企业非核心及时创新带来的国家核心竞争力的丧
失承担损失,需要承担的损失记为$C_2$。
我们将政府监管概率记为$x$, $0 < x < 1$, 企
业进行核心技术创新的概率记为$y$ $0 < y < 1$,以此构造的支付矩阵如表 1所示

\begin{tabular}{l|r|r}
             & 核心技术创新 $y$ & 非核心技术创新 $1-y$ \\
\hline
参与$x$      & $-A+b_1-C_1$ & $-C_1$ \\
\hline
不参与$1-x$  & 0 & -$C_2$ \\
\end{tabular}
\end{document}
